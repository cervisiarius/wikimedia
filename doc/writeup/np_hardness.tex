\documentclass[12pt,letterpaper]{article}
\usepackage[utf8]{inputenc}
\usepackage{amsmath}
\usepackage{amsfonts}
\usepackage{amssymb}
\usepackage{fullpage}

\begin{document}
\title{The Link Placement Problem}
\author{Bob West}
\maketitle

\section{Problem formulation}

We have:

- a hyperlink network $G=(V,E)$

- a set $P$ of navigation paths over this network

- model: after the navigation paths were collected, we introduce some previously non-existent links

- now, for each path $p=(p_1,\dots,p_n)$, we let the user revisit the path step by step, and for each triple $(p_i,p_{i+1},p_{i+2})$ along the path, the user has the option to choose the shortcut $(p_i,p_{i+2})$ if that shortcut is one of the newly introduced links; i.e., the user may choose to skip over $p_{i+1}$

- we assume that, for each node triple $(s,m,t)$ with $(s,t) \notin E$, there is a fixed probability $q(s,m,t)$ with which the user will choose the shortcut $(s,t)$ over the full triple $(s,m,t)$

- this way, we can compute the probability of being chosen for each shortcut link along each path

- consider a path $p=(p_1,\dots,p_n)$ and let $f^p_i$ be the probability that the (old) link $(p_i,p_{i+1})$ is clicked on path $p$, and $g^p_i$ the probability that the (new) link $(p_i,p_{i+2})$ is clicked

- define $x(s,t)$ as a binary variable indicating whether $(s,t)$ is one of the newly introduced links

- these two probabilities are defined recursively as follows:
\begin{eqnarray}
\begin{array}{llll}
f^p_i &=& \left(f^p_{i-1} + g^p_{i-2}\right) \, (1 - x(p_i,p_{i+2}) \, q(p_i,p_{i+1},p_{i+2})) & \mbox{for $i=1,\dots,n-2$} \\
g^p_i &=& \left(f^p_{i-1} + g^p_{i-2}\right) \, x(p_i,p_{i+2}) \, q(p_i,p_{i+1},p_{i+2}) & \mbox{for $i=1,\dots,n-2$} \\
f^p_{i} &=& f^p_{i-1} + g^p_{i-2} & \mbox{for $i=n-1,n$} \\
f^p_0 &=& 1 & \\
g^p_{-1} &=& g^p_0 \;\;=\;\; g^p_{n-1} \;\;=\;\; g^p_n \;\;=\;\; 0 &
\end{array}
\end{eqnarray}

- if $(p_i,p_{i+2})$ is not one of the newly introduced links, then $g^p_i=0$

- flow interpretation: one unit of flow is transmitted from $p_1$ to $p_n$

- the flow that enters each node must equal the flow that leaves it

- each newly introduced shortcut link $(s,t)$ may appear in several paths, and has a probability of being clicked on each of these paths

- the value of that probability depends on the entire set of new shortcut links

- by summing these probabilities for fixed $(s,t)$ across all paths, we obtain the number of clicks that shortcut is expected to receive

- so we can now phrase our optimization objective: given a budget of $K$ shortcut links that may be added, choose those shortcuts that would receive the maximum expected number of clicks

- more formally: we want to maximize $\sum_{p \in P} \sum_{i=1}^{|p|-2} g^p_i$

\subsection{Integer linear program}


\section{NP-hardness}

By reduction from HITTING SET.%
\footnote{Garey and Johnson, Problem SP8, p.~222}

\subsection{Definition of the LINK PLACEMENT problem}

\begin{itemize}
\item INSTANCE: Directed graph $G=(V,E)$, set of paths $P \subseteq V^*$, shortcut probabilities $Q=\{q_{st} : (s,t) \in V^2 \setminus E\}$, positive integer $K$.
\item QUESTION: Given a budget of $K$ shortcut links $(s_1,t_1), \dots, (s_K,t_K)$, what is the maximum cumulative flow over the shortcuts across all paths?
\end{itemize}

\subsection{Definition of the HITTING SET problem}

\begin{itemize}
\item INSTANCE: Collection $C$ of subsets of a finite set $Z$, positive integer $L \leq |Z|$.
\item QUESTION: Is there a subset $Z' \subseteq Z$ with $|Z'| \leq L$ such that $Z'$ contains at least one element from each subset in $C$?
\end{itemize}

Importantly, the problem remains NP-complete even if $|c| \leq 2$ for all $c \in C$.

\subsection{Reduction}

We need to reduce a HITTING SET instance $(C,Z,L)$ to a LINK PLACEMENT instance $(V,E,P,Q,K)$.
We assume w.l.o.g.\ that each $c \in C$ contains at most two elements.

First, we set $K=L$.

$V=\bigcup_{i=1}^{|Z|} \{s_i, m_i, t_i\}$.

For each $c \in C$: if $c=\{z_i,z_j\}$, add a path $(s_i,s_j,t_i,t_j)$ to $P$; if $c=\{z_i\}$, add a path $(s_i,m_i,t_i)$ to $P$.

$E$ is the set of edges induced by $P$.

$q_{st}=1$ for all $(s,t) \in V^2 \setminus E$.


\section{Approximate solutions}

\subsection{Submodularity}


\end{document}
